\section{Resumen}
Los evaluadores hacen splines y superficies que se basan en una base de Bézier (o Bernstein). Si se desea utilizar el evaluador para trazar curvas y superficies utilizando otras bases, se debe saber cómo convertir su base en una base Bezier. Además, al renderizar una superficie Bézier o parte de ella utilizando el evaluador, es necesario especificar el nivel de detalle de su subdivisión. La funcionalidad NURBS de  GLU es una interfaz de alto nivel: los procesos NURBS encapsulan una gran cantidad de código complejo,los procesos NURBS utilizan polígonos planos para el renderizado.

\section{Abstract}
Evaluators make splines and surfaces that are based on a Bézier (or Bernstein) basis. If you want to use the evaluator to draw curves and surfaces using other bases, you must know how to convert your base to a Bézier base. Also, when rendering a Bezier surface or part of it using the evaluator, it is necessary to specify the level of detail of its subdivision. GLU's NURBS functionality is a high-level interface: NURBS processes encapsulate a lot of complex code, NURBS processes use flat polygons for rendering.

\section{Introducción}
En el nivel más bajo, el hardware de gráficos dibuja puntos, líneas y polígonos, que suelen ser triángulos y cuadriláteros.Las curvas y las superficies suaves se dibujan aproximandolas con líneas, polígonos grandes  o pequeños. Sin embargo, muchas curvas y superficies útiles se pueden describir matemáticamente mediante una pequeña cantidad de parámetros, como algunos puntos de control. Registrar los 16 puntos de control de una superficie requiere mucho menos espacio que registrar 1000 triángulos  con  información vectorial normal en cada vértice. Además, los 1000 triángulos solo se aproximan a la superficie real, pero los puntos de control representan con precisión la superficie real. \\

Los evaluadores permiten especificar puntos en una curva o superficie (o parte de ella) simplemente usando puntos de control. La curva o la superficie se pueden mostrar con cualquier precisión. Además, los vectores normales se pueden calcular automáticamente para las superficies. Los puntos generados por el evaluador se pueden usar de muchas maneras: para dibujar puntos en ubicaciones de superficie, para dibujar una versión alámbrica de la superficie, para dibujar una superficie sombreada o textura.\\

El evaluador se puede utilizar para describir todas las splines o superficies de polinomios o polinomios racionales en cualquier grado. Esto incluye casi todas las splines y superficies de splines en uso hoy en día, incluidas B-splines, NURBS (B-Splines racionales no uniformes), curvas y superficies de Bézier y splines de Hermite. Dado que los evaluadores solo brindan descripciones de bajo nivel de puntos en curvas o superficies, a menudo se usan en bibliotecas de utilidades que brindan una interfaz de nivel superior para el programador. La función NURBS de GLU es una interfaz de alto nivel: Los procesos NURBS encapsulan una gran cantidad de código complejo. Gran parte del renderizado final se realiza con el evaluador, pero para ciertas condiciones (por ejemplo, recorte de curvas), los procesos NURBS utilizan polígonos planos para el renderizado.\\

Este capítulo contiene las siguientes secciones principales: Los evaluadores, en ella se explica cómo funcionan los evaluadores y cómo controlarlos utilizando los comandos de OpenGL y la interfaz NURBS de GLU, la cual describe las rutinas de GLU para crear superficies NURBS.
%%%%%%%%%%%%%%%%%%%%%%%%%%%%%%%%%%%%%%%%%%%%%%%%%%%%%%%%%%%%%%%%%%%%%%%%%%%%%
