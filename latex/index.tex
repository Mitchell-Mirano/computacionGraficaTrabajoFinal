\documentclass[]{article}
\usepackage[utf8]{inputenc}
\usepackage{listings}
\usepackage{xcolor}

\definecolor{codegreen}{rgb}{0,0.6,0}
\definecolor{codegray}{rgb}{0.5,0.5,0.5}
\definecolor{codepurple}{rgb}{0.58,0,0.82}
\definecolor{backcolour}{rgb}{0.95,0.95,0.92}

\lstdefinestyle{mystyle}{
    backgroundcolor=\color{backcolour},   
    commentstyle=\color{codegreen},
    keywordstyle=\color{magenta},
    numberstyle=\tiny\color{codegray},
    stringstyle=\color{codepurple},
    basicstyle=\ttfamily\footnotesize,
    breakatwhitespace=false,         
    breaklines=true,                 
    captionpos=b,                    
    keepspaces=true,                 
    numbers=left,                    
    numbersep=5pt,                  
    showspaces=false,                
    showstringspaces=false,
    showtabs=false,                  
    tabsize=2
}

\lstset{style=mystyle}

\begin{document}

\section*{Evaluadores bidimensionales}

En dos dimensiones, todo es similar al caso unidimensional, excepto que todos los comandos deben
Tenga en cuenta dos parámetros, U y V,.Los puntos, colores, normales o coordenadas de textura deben ser
suministrado sobre una superficie en lugar de una curva.Matemáticamente, la definición de un parche de superficie Bézier es
dada por


\begin{equation}
    S(u,v)=\sum_{i=0}^{n}\sum_{j=0}^{m} B_{i}^{n}(u) B_{j}^{m}(v)P_{ij}
\end{equation}

donde $P_{ij}$ son un conjunto de puntos de control $m*n$ y los $B_{i}$ son los mismos polinomios de Bernstein para uno
dimensión.Como antes, el $P_{ij}$ puede representar vértices, normales, colores o coordenadas de textura.


El procedimiento para usar evaluadores bidimensionales es similar al procedimiento para una dimensión.


\begin{enumerate}
    \item Defina los evaluadores (s) con glMap2*()
    \item Habilitarlos pasando el valor apropiado a glEnable().
    \item Invocarlos llamando a glEvalCoord2() entre un par glBegin() y glEnd() o por
    Especificar y luego aplicar una malla con glMapGrid2() y glEvalMesh2().
\end{enumerate}


\subsection*{Definición y evaluación de un evaluador bidimensional}


Use GLMAP2*() y GlevalCoord2*() para definir y luego invocar un evaluador bidimensional.
void glMap2{fd}(GLenum target, TYPEu1, TYPEu2, GLint ustride,
GLint uorder, TYPEv1, TYPEv2, GLint vstride,
GLint vorder, TYPE points);

El parámetro objetivo puede tener cualquiera de los valores en la Tabla 12-1, excepto que la cadena MAP1 es
reemplazado con MAP2.Como antes, estos valores también se usan con glEnable() para habilitar el
evaluador correspondiente.Los valores mínimos y máximos para U y V se proporcionan como U1, U2,
V1 y V2.Los parámetros de ustride y vstride indican el número de precisión única o doble
valores (según corresponda) entre la configuración independiente para estos valores, lo que permite a los usuarios seleccionar un
El subrectangle de control señala de una matriz mucho más grande.Por ejemplo, si los datos aparecen en el
forma.

GLfloat ctlpoints[100][100][3];

y desea usar el subconjunto 4x4 que comienza en cltpoints[20][30], elija Ustride para que sea 100*3 y
Vstride para ser 3. El punto de partida, puntos, debe establecerse en $\&ctlpoints[20][30][0]$.Finalmente, el
Los parámetros de pedido, Uorder y Vorder pueden ser diferentes, lo que permite parches cúbicos en uno
dirección y cuadrática en el otro, por ejemplo.


\begin{enumerate}
    \item void glEvalCoord2{fd}(TYPE u, TYPE v)
    \item void glEvalCoord2{fd}v(TYPE *values)
\end{enumerate}
 
 
Causa la evaluación de los mapas bidimensionales habilitados.Los argumentos U y V son los valores
(o un puntero a los valores U y V, en la versión vectorial del comando) para las coordenadas de dominio.
Si se habilita cualquiera de los evaluadores de vértices ($ GL_MAP2_VERTEX_3 $ o $ GL_MAP2_VERTEX_4 $),
entonces lo normal a la superficie se calcula analíticamente.Esta normal está asociada con el
vértice generado si la generación normal automática se ha habilitado pasando
$GL_AUTO_NORMAL$ a glEnable().Si está deshabilitado, el mapa normal habilitado habilitado correspondiente es
utilizado para producir una normal.Si no existe tal mapa, se usa la corriente actual.

Ejemplo 12-2 dibuja una superficie de estructura bézier utilizando evaluadores, como se muestra en la Figura 12-2.En esto
Ejemplo, la superficie se dibuja con nueve líneas curvas en cada dirección.Cada curva se dibuja como 30
segmentos.Para obtener todo el programa, agregue las rutinas Reshape () y Main () del ejemplo 12-1.

\lstinputlisting[language=C++]{codes/example12-2.cpp}

\subsection*{Definición de valores de coordenadas espaciados uniformemente en dos dimensiones}

En dos dimensiones, los comandos glMapGrid2*() y glEvalMesh2() son similares a los
Versiones unidimensionales, excepto que se deben incluir información de U y V.

\begin{enumerate}
    \item[-] void glMapGrid2{fd}(GLint nu, TYPEu1, TYPEu2,
    GLint nv, TYPEv1, TYPEv2);
    \item[-] void glEvalMesh2(GLenum mode, GLint i1, GLint i2, GLint j1, GLint j2);
\end{enumerate}
 

Define una cuadrícula de mapa bidimensional que va de U1 a U2 en pasos nu uniformemente espaciados, de V1 a
V2 en pasos NV (glMapGrid2*()), y luego aplica esta cuadrícula a todos los evaluadores habilitados
(glEvalMesh2()).La única diferencia significativa con las versiones unidimensionales de estos dos
Los comandos son que en glevalmesh2 () el parámetro de modo puede ser $GL\_FILL$ y $GL\_POINT$
o $GL_LINE$ $GL_FILL$ genera polígonos rellenos utilizando el primitivo de malla cuádruple.Declarado precisamente,
GlevalMesh2 () es casi equivalente a uno de los siguientes tres fragmentos de código.(Es casi
equivalente porque cuando yo es igual a nu o j a nv, el parámetro es exactamente igual a u2 o v2, no
a $u1+nu*(u2-u1)/nu$, que podría ser ligeramente diferente debido al error redondeado).


\begin{lstlisting}[language=C++]
    glBegin(GL_POINTS); /* mode == GL_POINT */
        for (i = nu1; i <= nu2; i++)
        for (j = nv1; j <= nv2; j++)
        glEvalCoord2(u1 + i*(u2-u1)/nu, v1+j*(v2-v1)/nv);
    glEnd();
    
\end{lstlisting}

o


\begin{lstlisting}[language=C++]

    for (i = nu1; i <= nu2; i++) { /* mode == GL_LINE */
    glBegin(GL_LINES);
        for (j = nv1; j <= nv2; j++)
        glEvalCoord2(u1 + i*(u2-u1)/nu, v1+j*(v2-v1)/nv);
    glEnd();
    }

    for (j = nv1; j <= nv2; j++) {
    glBegin(GL_LINES);
        for (i = nu1; i <= nu2; i++)
        glEvalCoord2(u1 + i*(u2-u1)/nu, v1+j*(v2-v1)/nv);
    glEnd();
    }

\end{lstlisting}

or

\begin{lstlisting}[language=C++]
    for (i = nu1; i < nu2; i++) { /* mode == GL_FILL */
        glBegin(GL_QUAD_STRIP);
        for (j = nv1; j <= nv2; j++) {
            glEvalCoord2(u1 + i*(u2-u1)/nu, v1+j*(v2-v1)/nv);
            glEvalCoord2(u1 + (i+1)*(u2-u1)/nu, v1+j*(v2-v1)/nv);
        glEnd();
    }
}
    
\end{lstlisting}

Ejemplo 12-3 muestra las diferencias necesarias para dibujar la misma superficie de Bézier que el ejemplo 12-2, pero
Usando glMapGrid2() y glEvalMesh2() para subdividir el dominio cuadrado en una cuadrícula uniforme de 8x8.Este
El programa también agrega iluminación y sombreado, como se muestra en la Figura 12-3.


\lstinputlisting[language=C++]{codes/example12-3.cpp}


\subsection*{Usar evaluadores para texturas}


Ejemplo 12-4 habilita dos evaluadores al mismo tiempo: el primero genera puntos tridimensionales en
La misma superficie de Bézier que el ejemplo 12-3, y la segunda genera coordenadas de textura.En este caso, el
Las coordenadas de textura son las mismas que las coordenadas U y V de la superficie, pero un parche especial de Bézier
debe crearse para hacer esto.
El parche plano se define sobre un cuadrado con esquinas en (0, 0), (0, 1), (1, 0) y (1, 1);genera (0, 0) en
esquina (0, 0), (0, 1) en la esquina (0, 1), y así sucesivamente.Ya que es de orden dos (grado lineal más uno), evaluando

Esta textura en el punto (U, V) genera coordenadas de textura (S, T).Está habilitado al mismo tiempo que el
Evaluador de vértice, por lo que ambos surtan efecto cuando se dibuja la superficie.(Ver "Placa 19" en el Apéndice I.) Si
Desea que la textura repita tres veces en cada dirección, cambie cada 1.0 en la matriz texpts[][][] a 3.0.
Dado que la textura se envuelve en este ejemplo, la superficie se representa con nueve copias del mapa de textura.



\lstinputlisting[language=C++]{codes/example12-4.cpp}



\end{document}